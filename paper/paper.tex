\documentclass[sigplan,screen]{acmart}

\AtBeginDocument{%
  \providecommand\BibTeX{{%
    Bib\TeX}}}

\setcopyright{acmlicensed}
\copyrightyear{2025}
\acmYear{2025}
\acmDOI{XXXXXXX.XXXXXXX}

\acmConference[Conference acronym 'XX]{Make sure to enter the correct
  conference title from your rights confirmation email}{June 03--05,
  2018}{Woodstock, NY}
\acmISBN{978-1-4503-XXXX-X/18/06}


\begin{document}

\title{Bluefin: a novel effect system for Haskell}

\author{Simon Peyton Jones}
\affiliation{%
  \institution{Epic Games}
  \city{Cambridge}
  \country{UK}
}
\email{spj@spjemail}

\author{Tom Ellis}
\affiliation{%
  \institution{Groq, Inc.}
  \city{Cambridge}
  \country{UK}
  }
\email{tom@tomsemail}

\begin{abstract}
  Bluefin is a novel effect system for Haskell.  It is based around
  the monadic type \texttt{Eff}, which is a newtype around Haskell's
  \texttt{IO} type.  Effects are accessed through value-level
  capabilities, and an \texttt{ST}-like use of the type system ensures
  that capabilities cannot escape their scope.  We motivate the design
  and compare it to existing effect systems, explaining how it differs
  and why.
\end{abstract}

\maketitle

\section{Introduction}

There are a number of desirable properties that an effect system might
have.

\subsection{Encapsulation}

Encapsulation is the property than we can handle one of the effects
used by an effectful operation, and the type of the result indicates
that the effect is no longer externally visible.  For example,
Haskell's \texttt{ST} encapsulates state references throught its
function \texttt{runST :: (forall s. ST s a) -> a}.  The rank-2 type
ensures that no state references leak into the result.

\subsection{Resource safety}

\subsection{State, exceptions and I/O}
\subsection{Non-determinism}

\section{Bluefin}

Bluefin is a novel effect system for Haskell.  Let's motivate the
design of Bluefin through a sequence of strawmen...

\section{Comparison to other effect systems}

\begin{itemize}
\item Just IO, and/or ReaderT IO
\item effectful
\item polysemy
\item transformers
\item heftia-effects
\end{itemize}

See Table \ref{tab:comparison}.  Column key:

\begin{itemize}
\item S: state
\item E: exceptions
\item I: I/O
\item NonDet: non-determinism
\item Res: resource safety
\item Enc: encapsulation
\end{itemize}

\begin{table}
  \caption{Effect systems}
  \label{tab:comparison}
  \begin{tabular}{lcccccc}
    \toprule
    Effect system & S & E & I & NonDet & Res & Enc \\
    \midrule
    Bluefin & Y & Y & Y & N & Y & Y\\
    Polysemy & Y & Y & Y & Y & N & Y\\
    IO & Y & Y & Y & N & Y & N\\
    ST & Y & N & N & N & ? & Y\\
  \bottomrule
\end{tabular}
\end{table}

\end{document}
\endinput
